\usepackage{amsfonts,amsmath,amssymb}
\def\R{\mathbb{R}}
\def\Q{\mathbb{Q}}
\def\N{\mathbb{N}}
\def\Z{\mathbb{Z}}
\def\C{\mathbb{C}}

\def\I{\mathord{\mathrm{i}}}
\def\E{\mathord{\mathrm{e}}}

\def\tg{\mathop{\mathrm{tg}}\nolimits}
\def\cotg{\mathop{\mathrm{cotg}}\nolimits}
\def\sgn{\mathop{\mathrm{sgn}}\nolimits}

\def\dx{\,\mathrm{d}x}
\def\dy{\,\mathrm{d}y}
\def\ds{\,\mathrm{d}s}
\def\dt{\,\mathrm{d}t}

\usetikzlibrary{calc, intersections, through}

\tikzset{help lines/.style= {lightgray, ultra thin}}


\makeatletter
\def\obrMsr{\@ifstar\@obrMsr\@@obrMsr}

\def\@@celacast#1.#2\koneccelecasti{#1}

\def\obrMsr@test@hvezdicka{*}

\newcommand\@obrMsr[2][]{%
\def\@obrMsr@opt{#1}%
\def\@obrMsr@hvezdicka{*}%
\@obrMsr@detekuj@carku #2,*:::
\next #2,
}

\newcommand\@@obrMsr[2][]{%
\def\@obrMsr@opt{#1}%
\let\@obrMsr@hvezdicka\empty%
\@obrMsr@detekuj@carku #2,*:::
\next #2,
}

\def\@obrMsr@detekuj@carku #1,#2:::{\def\pom@{#2}%
  \ifx\pom@\obrMsr@test@hvezdicka\let\next\@nacti@ctyri\else\let\next\@nacti@jeden\fi
}

\def\@nacti@ctyri #1,#2#3#4{\def\MsrXmin{#1}\def\MsrXmax{#2}\def\MsrYmin{#3}\def\MsrYmax{#4}\obrMsr@dokonci}
\def\@nacti@jeden #1,#2,#3,#4,{\def\MsrXmin{#1}\def\MsrXmax{#2}\def\MsrYmin{#3}\def\MsrYmax{#4}\obrMsr@dokonci}

\def\obrMsr@nacti@rozmery #1,#2;{%
\def\@@pom{#1}\ifx\@@pom\empty\def\@@pom{0.7\linewidth}\fi
\def\@@@pom{#2}\ifx\@@@pom\empty\let\@@@pom\@@pom\fi
\edef\temp{\toks0={\the\toks0 , x=\@@pom/(\MsrXmax-\MsrXmin), y=\@@@pom/(\MsrYmax-\MsrYmin)}}%
\temp
}

\newcommand\obrMsr@dokonci[2][]{\def\pom@{#1}%
\toks0{\begin{tikzpicture}[inner sep=3pt, }%
\edef\temp{\toks0={\the\toks0 \@obrMsr@opt}}%
\temp
\ifx\pom@\empty\else
\obrMsr@nacti@rozmery #1;
\fi
\edef\temp{\toks0={\the\toks0 ]}}%
\temp
\the\toks0
\ifx\@obrMsr@hvezdicka\obrMsr@test@hvezdicka 
\obrOsaY\obrOsaX
\obrZnackyX{\expandafter\@@celacast\MsrXmin.\koneccelecasti,...,\MsrXmax}%
\obrZnackyY{\expandafter\@@celacast \MsrYmin.\koneccelecasti,...,\MsrYmax}%
\obrClip 
\fi
#2
\end{tikzpicture}%
}


\def\obrOsaX{\def\obr@osa@sipka{->}\@ifstar\@obrOsaX\@@obrOsaX}
\def\obrOsax{\def\obr@osa@sipka{-}\@ifstar\@obrOsaX\@@obrOsaX}
\newcommand\@obrOsaX[1][right]{\draw[\obr@osa@sipka] (\MsrXmin,0) -- (\MsrXmax,0);}
\newcommand\@@obrOsaX[1][right]{\draw[\obr@osa@sipka] (\MsrXmin,0) -- (\MsrXmax,0) node[#1] {$x$};}
\def\obrOsaY{\def\obr@osa@sipka{->}\@ifstar\@obrOsaY\@@obrOsaY}
\def\obrOsay{\def\obr@osa@sipka{-}\@ifstar\@obrOsaY\@@obrOsaY}
\newcommand\@obrOsaY[1][above]{\draw[\obr@osa@sipka] (0,\MsrYmin) -- (0,\MsrYmax);}
\newcommand\@@obrOsaY[1][above]{\draw[\obr@osa@sipka] (0,\MsrYmin) -- (0,\MsrYmax) node[#1] {$y$};}
\newcommand\obrOsy{\obrOsaX\obrOsaY}
\newcommand\obrFce[2][]{\draw[red,thick,samples=100,domain=\MsrXmin:\MsrXmax,#1] plot(\x,{#2});}
\newcommand\obrFcePar[3][]{\draw[red,thick,samples=100,domain=-3:3,#1] plot({#2},{#3});}
\newcommand\obrFceY[2][]{\draw[red,thick,samples=100,domain=\MsrYmin:\MsrYmax,#1] plot({#2},\x);}
\newcommand\obrClip{\clip (\MsrXmin,\MsrYmin) rectangle (\MsrXmax,\MsrYmax);}
\newcommand\obrBod[3][]{\draw [thin, dashed, #1] (0,{#3}) -- ({#2},{#3}) -- ({#2},0);} 
\newcommand\obrAsX[2][]{\draw [thin, dashed, #1] (#2,\MsrYmin) -- (#2,\MsrYmax);} 
\newcommand\obrAsY[2][]{\draw [thin, dashed, #1] (\MsrXmin,#2) -- (\MsrXmax,#2);} 

\newcommand\obrNula[1][anchor = north west]{\draw (0,0) node[#1] {$0$};}

\newcommand\obrZnackyX[2][]{\foreach \x/\xtext in {#2} \draw[shift={({\x},0)},#1] (0pt,2pt) -- (0pt,-2pt);}
\newcommand\obrZnackyY[2][]{\foreach \y/\ytext in {#2} \draw[shift={(0,{\y})},#1] (-2pt,0pt) -- (2pt,0pt);}

\def\@@MsrDetectLeft #1left#2\NeverAppears{\def\msr@tempkey{#2}}
\def\@@MsrDetectAbove #1above#2\NeverAppears{\def\msr@tempkey{#2}}

\newcommand\obrPopisX[2][below]{%
\@@MsrDetectAbove #1above\NeverAppears
\ifx\msr@tempkey\empty \def\msr@path{(0pt,2pt) -- (0pt,-2pt)} \else \def\msr@path{(0pt,-2pt) -- (0pt,2pt)} \fi
\foreach \x/\xtext/\styl in {#2} {%
\ifx\xtext\empty\let\xtext\x\fi
\ifx\xtext\styl \toks0{node[#1] {$\xtext$}}\else \toks1{node[#1,}\toks2{] {$\xtext$}} \edef\act{\toks0{\the\toks1 \styl\the\toks2}}\act\fi
\draw[shift={({\x},0)}] \msr@path\the\toks0;}}

\newcommand\obrPopisY[2][right]{%
\@@MsrDetectLeft #1left\NeverAppears
\ifx\msr@tempkey\empty \def\msr@path{(-2pt,0pt) -- (2pt,0pt)} \else \def\msr@path{(2pt,0pt) -- (-2pt,0pt)} \fi
\foreach \y/\ytext/\styl in {#2} {%
\ifx\ytext\empty\let\ytext\y\fi
\ifx\ytext\styl \toks0{node[#1] {$\ytext$}}\else \toks1{node[#1,}\toks2{] {$\ytext$}} \edef\act{\toks0{\the\toks1 \styl\the\toks2}}\act\fi
\draw[shift={(0,{\y})}] \msr@path\the\toks0;}}

\def\obr@popis@nacti@B #1/#2/#3/#4;{\def\obr@popis@nacti@a{#1}\def\obr@popis@nacti@b{#2}\def\obr@popis@nacti@c{#3}%
\ifx\obr@popis@nacti@b\empty\def\obr@popis@nacti@b{below}\fi
\ifx\obr@popis@nacti@c\empty\let\obr@popis@nacti@c\obr@popis@nacti@a\fi
}
\def\obr@popis@nacti@above{above}
\def\obr@popis@nacti@below{below}

\newcommand\obr@popis@nacti[1][]{\edef\obr@popis@nacti@styl{#1}\obr@popis@nacti@B}

\def\@obrPopis#1{%
\obr@popis@nacti #1//;
\ifx\obr@popis@nacti@b\obr@popis@nacti@above
  \edef\temp{[above,\obrPopis@styl,\obr@popis@nacti@styl]}%
  \expandafter\obrPopisX\temp{\obr@popis@nacti@a/\obr@popis@nacti@c}%
\else
\ifx\obr@popis@nacti@b\obr@popis@nacti@below
  \edef\temp{[below,\obrPopis@styl,\obr@popis@nacti@styl]}%
  \expandafter\obrPopisX\temp{\obr@popis@nacti@a/\obr@popis@nacti@c}%
\else
\edef\temp{[\obr@popis@nacti@b,\obrPopis@styl,\obr@popis@nacti@styl]}%
\expandafter\obrPopisY\temp{\obr@popis@nacti@a/\obr@popis@nacti@c}%
\fi
\fi
}

\newcommand\obrPopis[2][]{\edef\obrPopis@styl{#1}\foreach \cosi in {#2} {\expandafter\@obrPopis\expandafter{\cosi}}\let\obrPopis@styl\relax}

\def\msr@letter@o{o}
\def\msr@letter@n{n}
\def\msr@letter@u{u}

\def\msr@typ@save#1#2;{\def\temp{#1}%
\def\msr@int@sipka{-}
\def\msr@int@shorten{}
\ifx\temp\msr@letter@o\let\msr@obr@vlevo\draw\fi
\ifx\temp\msr@letter@u\let\msr@obr@vlevo\fill\fi
\ifx\temp\msr@letter@n\let\msr@obr@vlevo\msr@letter@n \def\msr@int@sipka{<-}\else \def\msr@int@shorten{shorten <=2pt,}\fi
\def\temp{#2}%
\ifx\temp\empty{\PackageError{msr-spolecne}{Nepodarilo se rozpoznat typy pro konec intervalu u makra obrInterval}{}}\fi%
\ifx\temp\msr@letter@o\let\msr@obr@vpravo\draw\fi
\ifx\temp\msr@letter@u\let\msr@obr@vpravo\fill\fi
\ifx\temp\msr@letter@n\let\msr@obr@vpravo\msr@letter@n \edef\msr@int@sipka{\msr@int@sipka>}\else \edef\msr@int@shorten{\msr@int@shorten,shorten >=2pt,}\fi
}

\newcommand{\obrInterval}[1][red]{\def\msr@interval@opt{#1}\interval@}
\define@key{obr}{y}[0]{\def\msr@obor@y{#1}}
\define@key{obr}{typ}[oo]{\msr@typ@save #1;}
\newcommand{\interval@}[3][]{\def\msr@obor@y{16pt}\msr@typ@save uu;
{\clip (\MsrXmin,\MsrYmin) rectangle (\MsrXmax,\MsrYmax);
\setkeys{obr}{#1}%
\edef\@msr@temp{thick, barva1, \msr@interval@opt}%
% prazde krouzky, je-li potreba
\ifx\msr@obr@vlevo\msr@letter@n\else
  \expandafter\draw\expandafter[\@msr@temp] (#2,\msr@obor@y) circle (2pt);
  \expandafter\draw\expandafter[\@msr@temp,thin,dashed,shorten >=2pt,shorten <=2pt]   (#2,\msr@obor@y) --  (#2,0);
\fi
\ifx\msr@obr@vpravo\msr@letter@n\else
  \expandafter\draw\expandafter[\@msr@temp] (#3,\msr@obor@y) circle (2pt);
  \expandafter\draw\expandafter[\@msr@temp,thin,dashed,shorten >=2pt,shorten <=2pt]   (#3,\msr@obor@y) --  (#3,0);
\fi
% plne krouzky, je-li potreba
\ifx\msr@obr@vlevo\fill\expandafter\fill\expandafter[\@msr@temp] (#2,\msr@obor@y) circle (2pt);\fi
\ifx\msr@obr@vpravo\fill\expandafter\fill\expandafter[\@msr@temp] (#3,\msr@obor@y) circle (2pt);\fi
\edef\@msr@temp{very thick, barva1, \msr@int@shorten, \msr@int@sipka, \msr@interval@opt}%
\def\msr@odkud@kreslit{#2} \ifx\msr@obr@vlevo\msr@letter@n \let\msr@odkud@kreslit\MsrXmin\fi
\def\msr@kam@kreslit{#3} \ifx\msr@obr@vpravo\msr@letter@n \let\msr@kam@kreslit\MsrXmax\fi
\expandafter\draw\expandafter[\@msr@temp] (\msr@odkud@kreslit,\msr@obor@y) -- (\msr@kam@kreslit,\msr@obor@y);
}}


\define@key{obruhel}{polomer}[6mm]{\def\msr@obruhel@polomer{#1}}
\define@key{obruhel}{styl}[]{\def\msr@obruhel@styl{#1}}
\define@key{obruhel}{fill}[vypln1]{\def\msr@obruhel@fill{#1}}
\define@key{obruhel}{znacka}[]{\def\msr@obruhel@znacka{#1}}
\define@key{obruhel}{polomerz}[0.65*(\msr@obruhel@polomer)]{\def\msr@obruhel@polomer@znacka{#1}}
\define@key{obruhel}{stylz}[]{\def\msr@obruhel@stylznacky{#1}}

\newtoks\@msr@obruhel@toks
\newdimen\@msrdimen@
\def\obrUhelpolomer{6mm}
\newcommand\obrUhel[4][]{
\def\msr@obruhel@polomer{\obrUhelpolomer}
\def\msr@obruhel@styl{}
\def\msr@obruhel@fill{}
\def\msr@obruhel@stylznacky{}
\def\msr@obruhel@znacka{}
\def\msr@obruhel@polomer@znacka{}
\setkeys{obruhel}{#1}
\ifx\msr@obruhel@fill\empty\else
  \expandafter\fill\expandafter[\msr@obruhel@fill] #2 -- ++ ({\msr@obruhel@polomer*cos(#3)},{\msr@obruhel@polomer*sin(#3)}) arc (#3:#4:\msr@obruhel@polomer) -- cycle;
\fi
\expandafter\draw\expandafter[\msr@obruhel@styl] #2 ++ ({\msr@obruhel@polomer*cos(#3)},{\msr@obruhel@polomer*sin(#3)}) arc (#3:#4:\msr@obruhel@polomer);
\ifx\msr@obruhel@znacka\empty\else
  \ifx\msr@obruhel@polomer@znacka\empty
     \setbox0=\hbox{$\msr@obruhel@znacka$}
     \@msrdimen@=\the\wd0
     \advance\@msrdimen@ by \the\ht0
     \divide\@msrdimen@ by 2
     \def\msr@obruhel@polomer@znacka{\msr@obruhel@polomer+\the\@msrdimen@}
  \fi
  \@msr@obruhel@toks{\msr@obruhel@znacka}
  \edef\msr@temp{#2 ++ ({(\msr@obruhel@polomer@znacka)*cos(((#3)+(#4))/2)},{(\msr@obruhel@polomer@znacka)*sin(((#3)+(#4))/2)}) node [\msr@obruhel@stylznacky] {$\the\@msr@obruhel@toks$}}
  \expandafter\draw \msr@temp;
\fi
}


\newcommand{\obrUhelB}{\@ifstar
                     \@obrUhelB%
                     \@obrUhelBstar%
}


\newcommand\@obrUhelB[3][8mm]{\def\msr@smer{left}\def\msr@savea{#1}\def\msr@saveb{#2}\def\msr@savec{#3}\@@obrUhelB}
\newcommand\@obrUhelBstar[3][8mm]{\def\msr@smer{right}\def\msr@savea{#1}\def\msr@saveb{#2}\def\msr@savec{#3}\@@obrUhelB}


\newcommand\@@obrUhelB[2][ ]{\def\msr@saved{#1}\def\msr@savee{#2}% 
  \draw ($(\msr@savec)!\msr@savea!(\msr@saveb)$) to[bend \msr@smer] ($(\msr@savec)!\msr@savea!(\msr@savee)$);
  \let\save@default@msr@poloha@znacky\@msr@poloha@znacky \let\@msr@poloha@znacky\msr@savea
  \obrUhelBZnacka{\msr@saveb}{\msr@savec}{\msr@savee}{$\msr@saved$}\let\@msr@poloha@znacky\save@default@msr@poloha@znacky
}


\newcommand\obrUhelBZnacka[5][]{\coordinate (pom1) at ($(#3)!\@msr@poloha@znacky!(#2)$); \coordinate (pom2) at ($(#3)!\@msr@poloha@znacky!(#4)$); 
  \draw ($0.333*(#3)+0.333*(pom1)+0.333*(pom2)$) node [#1] {#5};
}
\def\@msr@poloha@znacky{8mm}


\newcommand{\obrUhelPravy}{\@ifstar
                     \@obrUhelPravystar%
                     \@obrUhelPravy%
}

\newcommand\@obrUhelPravy[4][5mm]{
  \draw (#3) ($(#3)!#1!(#2)$) to[bend left] ($(#3)!#1!(#4)$) (#3) -- cycle;
  \coordinate (pom1) at ($(#3)!#1!(#2)$); \coordinate (pom2) at ($(#3)!#1!(#4)$); 
  \draw[fill] ($0.333*(#3)+0.333*(pom1)+0.333*(pom2)$) circle (0.5pt);
}

\newcommand\@obrUhelPravystar[4][5mm]{\@obrUhelPravy[#1]{#4}{#3}{#2}}

\newcommand\obrBodNaOseUhlu[5][6cm]{\coordinate (#2) at ($($(#3)!#1!(#5)$)!0.5!($(#3)!#1!(#4)$)$);}
% \obrBodNaOseUhlu[2cm]{A}{B}{C}{D} 
% na ose uhlu CBD vytvorime bod A tak, ze rozpulime spojnici body na
% ramenech uhlu ve vzdalenosti 2 cm od vrcholu.

\newcommand\obrPataKolmice[4]{\coordinate (#1) at ($(#3)!(#2)!(#4)$);}
% \obrPataKolmice{A}{B}{C}{D}
% Z bodu C spustime kolmici na usecku BD, prusecik pojmenujeme A

\def\obrTeziste#1#2#3#4{\coordinate (#1) at ($1/3*(#2)+1/3*(#3)+1/3*(#4)$);}
% \obrTeziste {A}{B}{C}{D}
% Bod A se umisti do teziste trojuhelniku BCD

\newcommand\obrKrizek[4][1.5pt]{%
\draw let \p1=(#2) in (\x1-#1,\y1-#1)--(\x1+#1,\y1+#1);
\draw let \p1=(#2) in (\x1+#1,\y1-#1)--(\x1-#1,\y1+#1);
\path (#2) node [#3] {${#4}$};
}

\def\obrStredOpsane#1#2#3#4{\path let \p1=($0.01*(#2)$), \p2=($0.01*(#3)$), \p3=($0.01*(#4)$), 
    \n1={veclen(\x2-\x3,\y2-\y3)}, \n2={veclen(\x1-\x3,\y1-\y3)}, \n3={veclen(\x2-\x1,\y2-\y1)}
    in 
    \pgfextra{%
      \pgfmathsetmacro{\Dres}{2*((\x2-\x1)*(\y3-\y1)-(\y2-\y1)*(\x3-\x1))}
      \pgfmathsetmacro{\Xres}{(( (\y3-\y1) * ((\x2-\x1)^2+(\y2-\y1)^2)   -  (\y2-\y1) * ( (\x3-\x1)^2 + (\y3-\y1)^2 ))/\Dres ) + \x1}
      \pgfmathsetmacro{\Yres}{(( - (\x3-\x1) * ((\x2-\x1)^2+(\y2-\y1)^2)   +  (\x2-\x1) * ( (\x3-\x1)^2 + (\y3-\y1)^2 ))/\Dres) + \y1}
    } ($100*(\Xres pt,\Yres pt)$) coordinate (#1);}
% \obrStredOpsane SABC
% do stredu kruznice opsane trojuhleniku ABC se umisti bod S

\def\obrStredVepsane#1#2#3#4{\path let \p1=($0.01*(#2)$), \p2=($0.01*(#3)$), \p3=($0.01*(#4)$), 
  \n1={veclen(\x2-\x3,\y2-\y3)}, \n2={veclen(\x1-\x3,\y1-\y3)}, \n3={veclen(\x2-\x1,\y2-\y1)},
  \n4={(\n1+\n2+\n3)}
  in 
  \pgfextra{%
    \pgfmathsetmacro{\Xres}{(\n1*\x1+\n2*\x2+\n3*\x3)/(\n4)}
    \pgfmathsetmacro{\Yres}{(\n1*\y1+\n2*\y2+\n3*\y3)/(\n4)}
  } ($100*(\Xres pt,\Yres pt)$) coordinate (#1);}
% \obrStredVepsane SABC
% do stredu kruznice vepsane trojuhleniku ABC se umisti bod S


% fixed exp function: http://tex.stackexchange.com/questions/31775/is-plotting-exponential-graphs-a-known-source-of-bugs-in-tikz
%

\let\pgfmath@function@exp\relax % undefine old exp function
\pgfmathdeclarefunction{exp}{1}{%   
  \begingroup
    \pgfmath@xc=#1pt\relax
    \pgfmath@yc=#1pt\relax
    \ifdim\pgfmath@xc<-9pt
      \pgfmath@x=1sp\relax
    \else
      \ifdim\pgfmath@xc<0pt
        \pgfmath@xc=-\pgfmath@xc
      \fi
      \pgfmath@x=1pt\relax
      \pgfmath@xa=1pt\relax
      \pgfmath@xb=\pgfmath@x
      \pgfmathloop%
        \divide\pgfmath@xa by\pgfmathcounter
        \pgfmath@xa=\pgfmath@tonumber\pgfmath@xc\pgfmath@xa%
        \advance\pgfmath@x by\pgfmath@xa
      \ifdim\pgfmath@x=\pgfmath@xb
      \else
        \pgfmath@xb=\pgfmath@x
      \repeatpgfmathloop%
      \ifdim\pgfmath@yc<0pt
        \pgfmathreciprocal@{\pgfmath@tonumber\pgfmath@x}%
        \pgfmath@x=\pgfmathresult pt\relax
      \fi
    \fi
    \pgfmath@returnone\pgfmath@x%
  \endgroup
}


\colorlet{barva1}{red}
\colorlet{barva2}{blue}
\colorlet{barva3}{black!80}
\colorlet{barva4}{green!60!black}
\colorlet{vypln1}{yellow!30}
\colorlet{vypln2}{red!20}
\colorlet{vypln3}{green!50!black!30}



% makra pro tabulky

\def\polozka@Star{\let\msr@obalka@odpovedi\ensuremath\polozka@Both}
\def\polozka@NoStar{\let\msr@obalka@odpovedi\relax\polozka@Both}

\let\polozkahrule\hrule
\def\polozka@Both#1\odpovedi#2\par{\item\qquad\begin{minipage}{\MsrTabulkaB}
      \msr@obalka@odpovedi{#1}
    \end{minipage}\hfill
    \begin{minipage}{\MsrTabulkaC}\ifx\ObalkaOdpovedi\relax\else\let\msr@obalka@odpovedi\ObalkaOdpovedi\fi\odpovedi{#2}
    \end{minipage}\vspace*{\MsrTabulkaA}{\color[rgb]{0.8,0.8,0.8} \polozkahrule}}
 
\newcommand\MsrTabulka[3][0pt]{%
\graylettersOff
\def\MsrTabulkaA{#1}%
\def\MsrTabulkaB{#2}%
\def\MsrTabulkaC{#3}%
\pocetsloupcu{4}%
\def\polozka{\@ifstar\polozka@Star\polozka@NoStar}%
}

\def\MsrTabulkaZahlavi#1#2#3{\leavevmode\hbox{\hspace*{#1}%
\foreach \polozka in {#3} {\rlap{\polozka}\hspace{#2}}}}


\def\seda{\color[rgb]{0.8,0.8,0.8}}
\InputIfFileExists{msr-local.tex}{}{}

\newcommand\napoveda[2][0.6\linewidth]{{\par{\begin{minipage}[t]{#1}\footnotetext{\textit{N�vod:} #2\par}\end{minipage}}}}

\def\zdroj#1{{\color[rgb]{0.5,0.5,0.5}\footnotesize Zdroj: #1}}
\makeatother


%% viz http://tex.stackexchange.com/questions/2003/the-trestle-problem-how-to-avoid-in-the-output/4381#4381
\mathchardef\lt=\mathcode`\<
\mathchardef\gt=\mathcode`\>
%\mathchardef\strednik=\mathcode`\;
\let\zavorkal(
\let\zavorkap)
{\catcode`\<=\active
 \catcode`\>=\active
% \catcode`\;=\active
 \catcode`\(=\active
 \catcode`\)=\active
 \global\def<{\left\langle}
 \global\def>{\right\rangle}
 \global\def({\left\zavorkal}
 \global\def){\right\zavorkap}
% \global\def;{\strednik\ \ignorespaces}
}

\def\AktivniZavorky{
\mathcode`\<="8000
\mathcode`\>="8000
\mathcode`\(="8000
\mathcode`\)="8000
}

%\mathcode`\;="8000

\let\setminus\smallsetminus
\makeatletter
\renewcommand*\env@matrix[1][*\c@MaxMatrixCols c]{%
  \hskip -\arraycolsep
  \let\@ifnextchar\new@ifnextchar
  \array{#1}}
\makeatother


\def\textime#1{}
\def\tiket#1{}
\def\msr#1{}

% komplexni konjugovanost: %http://tex.stackexchange.com/questions/22100/the-bar-and-overline-commands
\newcommand{\overbar}[1]{\mkern1.5mu\overline{\mkern-1.5mu#1\mkern-1.5mu}\mkern 1.5mu}

\makeatletter
% Upravene makro z latex.ltx (doplneny \rightskip)
\def\@parboxrestore{\@arrayparboxrestore\let\\\@normalcr\rightskip 0 pt plus 1em\relax}
\makeatother


%%% 31.8.2013 Makra pro vybarveni oblasti mezi dvema funkcemi
\makeatletter


\newtoks\msr@vyplntoks
\newtoks\msr@vyplntoks@temp

\newif\ifmsr@vypln@A \msr@vypln@Afalse
\newif\ifmsr@vypln@B \msr@vypln@Bfalse

\newcommand\save@vypln[1][NONE]{
\expandafter\gdef\csname msr@vyplnopt@\vypln@co\endcsname{#1}
\def\temp{#1}%
\ifx\temp\msr@NONE \csname msr@vypln@\vypln@co true\endcsname\else \csname msr@vypln@\vypln@co false\endcsname\fi
\save@vypln@b
}

\def\save@vypln@b#1;;--;;{\expandafter\gdef\csname msr@vypln@\vypln@co\endcsname{#1}}

% Prvni parametr je ulozen do prikazu \msr@vypln@A, pokud zacina
% hrnatou zavorkou, je obsah hranate zavorky v \msr@vyplnopt@A a
% zbytek v \msr@vypln@A. Podobne i dalsi tri parametry a prismenka B C
% a D.


\def\pridej@msr@vyplntoks#1{\msr@vyplntoks@temp{#1}\edef\act{\global\msr@vyplntoks{\the\msr@vyplntoks \the\msr@vyplntoks@temp}}\act}


\def\msr@none{none}
\def\msr@NONE{NONE}

\def\obrVypln{\@ifstar\obrVypln@Star\obrVypln@NoStar}
\def\obrVypln@Star{\the\msr@vyplntoks \global\msr@vyplntoks={}}

\newcommand\obrVypln@NoStar[5][]
{%
\msr@vyplntoks{}%
\def\vypln@co{A}\save@vypln #2;;--;;%
\def\vypln@co{B}\save@vypln #3;;--;;%
\def\vypln@co{C}\save@vypln #4;;--;;%
\def\vypln@co{D}\save@vypln #5;;--;;%
{\let\x\relax
\xdef\pom{[vypln1,#1] plot[domain=\msr@vypln@A:\msr@vypln@B] (\x,{\msr@vypln@C}) -- plot [domain=\msr@vypln@B:\msr@vypln@A] (\x,{\msr@vypln@D});}%
}%
\expandafter\fill\pom

\ifx\msr@vyplnopt@C\msr@NONE\def\msr@vyplnopt@C{}\fi
\ifx\msr@vyplnopt@D\msr@NONE\def\msr@vyplnopt@D{}\fi
\ifx\msr@vyplnopt@C\msr@none\else \expandafter\pridej@msr@vyplntoks\expandafter{\expandafter\obrFce\expandafter[\msr@vyplnopt@C]{\msr@vypln@C}} \fi
\ifx\msr@vyplnopt@D\msr@none\else \edef\msr@vyplnopt@D{blue,\msr@vyplnopt@D}\expandafter\pridej@msr@vyplntoks\expandafter{\expandafter\obrFce\expandafter[\msr@vyplnopt@D]{\msr@vypln@D}} \fi

%\ifx\msr@vyplnopt@A\empty\else
\ifmsr@vypln@A\else
{
\xdef\x{\msr@vypln@A}
\xdef\msr@vypln@CC{\msr@vypln@C}
\xdef\msr@vypln@DD{\msr@vypln@D}
\pridej@msr@vyplntoks{\draw[thin, dashed, }
\expandafter\pridej@msr@vyplntoks\expandafter{\msr@vyplnopt@A] (\msr@vypln@A,}
\expandafter\pridej@msr@vyplntoks\expandafter{\expandafter{\msr@vypln@CC}) -- (\msr@vypln@A,}
\expandafter\pridej@msr@vyplntoks\expandafter{\expandafter{\msr@vypln@DD});}
}
\fi

%\ifx\msr@vyplnopt@B\empty\else
\ifmsr@vypln@B\else
{
\xdef\x{\msr@vypln@B}
\xdef\msr@vypln@CC{\msr@vypln@C}
\xdef\msr@vypln@DD{\msr@vypln@D}
\pridej@msr@vyplntoks{\draw[thin, dashed, }
\expandafter\pridej@msr@vyplntoks\expandafter{\msr@vyplnopt@B] (\msr@vypln@B,}
\expandafter\pridej@msr@vyplntoks\expandafter{\expandafter{\msr@vypln@CC}) -- (\msr@vypln@B,}
\expandafter\pridej@msr@vyplntoks\expandafter{\expandafter{\msr@vypln@DD});}
}
\fi

}


% tikz3d
\usetikzlibrary{3d}
\usepgflibrary{decorations.pathmorphing}
\usetikzlibrary{decorations.pathmorphing}

\tikzoption{canvas is xy plane at z}[]{%
  \def\tikz@plane@origin{\pgfpointxyz{0}{0}{#1}}%
  \def\tikz@plane@x{\pgfpointxyz{1}{0}{#1}}%
  \def\tikz@plane@y{\pgfpointxyz{0}{1}{#1}}%
  \tikz@canvas@is@plane
}
\def\msrdecorate{decorate [decoration={random steps,segment length=2mm,amplitude=3pt},rounded corners=2pt]}


\def\msrzpracujIIID#1=#2;{\expandafter\gdef\csname msr#1IIID\endcsname{#2}}

\newif\ifMsrIIIDkreslitOsy
\MsrIIIDkreslitOsytrue

\newcommand\MsrIIID[2][]{
%nastaveni parametru - defaultni hodnoty
\gdef\msrdelkaIIID{0.9cm}\gdef\msrzvetseniIIID{0.85}\gdef\msrsirkaIIID{4cm}\gdef\msrvyskaIIID{5cm}
\gdef\msrXmaxIIID{5}\gdef\msrYmaxIIID{5}\gdef\msrZmaxIIID{5}
%nstaveni paremetru podle volitelneho parametru prikazu
\foreach \polozka in {#1} {\expandafter\msrzpracujIIID\polozka;}
%sestaveni obrazku
\begin{tikzpicture}[scale=\msrzvetseniIIID,x = { ({-0.5*cos(45)*\msrdelkaIIID},{-0.5*sin(45)*\msrdelkaIIID})},
                   y = { (\msrdelkaIIID,0cm)},
                   z = { (0cm,\msrdelkaIIID)}
]
\ifMsrIIIDkreslitOsy
\draw[->] (0,0,0)--(0,0,\msrZmaxIIID) node[above] {$z$};
\draw[->] (0,0,0)--(0,\msrYmaxIIID,0) node[right] {$y$};
\draw[->] (0,0,0)--(\msrXmaxIIID,0,0) node[below left] {$x$};
\fi
#2
\end{tikzpicture}
}

\def\obrZnackyIIID#1#2#3{
\begin{scope}
  \clip (0.9*\msrXmaxIIID,0,0) rectangle (0,0.9*\msrYmaxIIID,0.9*\msrZmaxIIID);
  \begin{scope}[canvas is yz plane at x=0]
    \obrZnackyX{1,...,#1}
    \obrZnackyY{1,...,#2}
  \end{scope}
  \begin{scope}[canvas is xz plane at y=0]
    \obrZnackyX[rotate=-45,scale=2]{1,...,#3}
  \end{scope}
\end{scope}

}

\def\popisekIIIDxdefault{below right}
\def\popisekIIIDydefault{below}
\def\popisekIIIDzdefault{left}

\def\msrtestX{x}
\def\msrtestY{y}
\def\msrtestZ{z}

\def\PopisekIIID#1=#2;{\def\temp{#1}
\ifx\temp\msrtestX \path(#2,0,0) node[below right] {$#2$}; \fi
\ifx\temp\msrtestY \path(0,#2,0) node[below] {$#2$}; \fi
\ifx\temp\msrtestZ \path(0,0,#2) node[left] {$#2$}; \fi
}

\def\PopiskyIIID#1{\foreach \pop in {#1} {\expandafter\PopisekIIID \pop;}}

\pgfmathsetseed{1500}

\makeatother


%\def\Vec#1#2{\overrightarrow{#1#2}}

\usepackage{esvect}

\AtBeginDocument{\pdfmapfile{+esvect.map}}
%\def\vec#1{\vv{#1}}
\def\Vec#1#2{\vv{#1#2}}

\def\primka{\mathord{\leftrightarrow}}
\def\O{\mathcal{O}}
\def\T{\mathcal{T}}
\def\S{\mathcal{S}}
\def\RR{\mathcal{R}}


%%% Local Variables: 
%%% coding: cp1250
%%% End: 
